\documentclass[german]{article}
\usepackage[T1]{fontenc}
\usepackage[utf8]{inputenc}
\usepackage{lmodern}
\usepackage[a4paper]{geometry}
\usepackage{babel}

\providecommand{\tightlist}{%
	\setlength{\itemsep}{0pt}\setlength{\parskip}{0pt}}

\begin{document}

\section*{Satzung}\label{satzung}

\emph{des Netz39 e.V. - Hackerspace Magdeburg}

in der Fassung vom 19.10.2016

\subsection*{Präambel}\label{pruxe4ambel}

Im folgenden finden sich Satzungen und Ordnungen des Vereins Netz39 auf
dem Stand vom 19.10.2016. In diesem Dokument wird der Einfachheit halber
ein generisches Maskulinum (z.B. ``ein Mitglied'') verwendet. Damit sind
jedoch stets alle Geschlechter gemeint. An Stellen, an denen
schriftliche Kommunikation gefordert wird, ist E-Mail stets mit
eingeschlossen.

\subsection*{Vereinssatzung}\label{vereinssatzung}

\subsubsection*{§1 Name und Sitz des Vereins;
Geschäftsjahr}\label{name-und-sitz-des-vereins-geschuxe4ftsjahr}

\begin{enumerate}
\def\labelenumi{\arabic{enumi}.}
\item
  Der Verein führt den Namen ``Netz39'' -- im Folgenden ``Verein''
  genannt.
\item
  Der Verein hat seinen Sitz in Magdeburg. Er soll in das
  Vereinsregister eingetragen werden und führt anschließend den Zusatz
  ``e.V.''.
\item
  Das Geschäftsjahr ist das Kalenderjahr.
\end{enumerate}

\subsubsection*{§2 Vereinszweck}\label{vereinszweck}

\begin{enumerate}
\def\labelenumi{\arabic{enumi}.}
\item
  Der Zweck des Vereins ist die Förderung von Wissenschaft, Forschung,
  Kunst, Kultur, Meinungs- und Wissensaustausch. Insbesondere sollen
  Bereiche der Informations. und Kommunikationsmedien,
  informationsverarbeitende Technologien, deren angrenzende Fachgebiete,
  handwerkliches Geschick, autodidaktisches Lernen, Erwachsenen- und
  Jugendbildung gefördert werden. Auf diese Weise sollen Kultur,
  Computerkunst, Bildung und Wissenschaft in neuen und bestehenden
  Formen ermöglicht werden. Netz39 schafft einen Anlaufpunkt für den
  technischen, gesellschaftlichen und kulturellen Austausch im Bereich
  informationsverarbeitender Technologien.
\item
  Für die Erfüllung dieser satzungsgemäßen Zwecke sollen geeignete
  Mittel durch Beiträge/Umlagen, Spenden, Zuschüsse und sonstige
  Zuwendungen eingesetzt werden.
\item
  Der Verein verfolgt ausschließlich und unmittelbar gemeinnützige
  Zwecke im Sinne des Abschnitts ``Steuerbegünstigte Zwecke'' der
  Abgabenordnung 1977 (§5 ff AO) in der jeweils gültigen Fassung. Der
  Verein ist selbstlos tätig. Er verfolgt nicht in erster Linie
  eigenwirtschaftliche Zwecke.
\item
  Mittel des Vereins dürfen nur für die satzungsgemäßen Zwecke verwendet
  werden.
\item
  Die Mitglieder erhalten in ihrer Eigenschaft als Mitglieder keine
  Zuwendungen aus Mitteln des Vereins. Es darf keine Person durch
  Ausgaben, die den Zwecken des Vereins fremd sind, oder durch
  unverhältnismäßig hohe Vergütungen begünstigt werden.
\end{enumerate}

\subsubsection*{§3 Mitgliedschaft}\label{mitgliedschaft}

\begin{enumerate}
\def\labelenumi{\arabic{enumi}.}
\item
  Der Verein besteht aus aktiven Mitgliedern (§3 Abs. 2) und
  Fördermitgliedern (§3 Abs. 3), die bereit sind, sich für die
  Erreichung der Vereinszwecke einzusetzen. Die Beitrittserklärung
  erfolgt schriftlich gemäß §4 gegenüber dem Vorstand.
\item
  Die aktive Mitgliedschaft kann von jeder natürlichen Person erworben
  werden, die sich zum Vereinszweck bekennt und durch aktive Mitarbeit
  einen regelmäßigen Beitrag leistet.
\item
  Die Fördermitgliedschaft kann von natürlichen und juristischen Person
  erworben werden, die den Verein bei der Erreichung seines
  Vereinszwecks unterstützen will.
\item
  Über die Aufnahme als Mitglied entscheidet der Vorstand. Will der
  Vorstand einen Aufnahmeantrag ablehnen, so legt er ihn der nächsten
  Mitgliederversammlung vor. Diese entscheidet endgültig.
\end{enumerate}

\subsubsection*{§4 Beginn und Ende der
Mitgliedschaft}\label{beginn-und-ende-der-mitgliedschaft}

\begin{enumerate}
\def\labelenumi{\arabic{enumi}.}
\item
  Die Mitgliedschaft muss gegenüber dem Vorstand schriftlich beantragt
  werden. Über den schriftlichen Aufnahmeantrag wird entsprechend §3
  entschieden. Der Vorstand ist nicht verpflichtet, dem Antragsteller
  Ablehnungsgründe mitzuteilen.
\item
  Die Mitgliedschaft beginnt mit der Aushändigung einer entsprechenden
  Bestätigung durch ein Vorstandsmitglied.
\item
  Die Mitgliedschaft endet
\end{enumerate}

\begin{itemize}
\tightlist
\item
  durch freiwillige Beendigung mit zweiwöchiger Frist zum Ende des
  Quartals durch schriftliche Erklärung gegenüber dem Vorstand;
\item
  bei natürlichen Personen durch Tod;
\item
  bei juristischen Personen mit Abschluss der Liquidation;
\item
  bei Eintritt der Insolvenz des Vereins;
\item
  durch Ausschluss gemäß §6.
\end{itemize}

\begin{enumerate}
\def\labelenumi{\arabic{enumi}.}
\setcounter{enumi}{3}
\item
  Bei Beendigung der Mitgliedschaft, gleich aus welchem Grund, erlöschen
  alle Ansprüche aus dem Mitgliedsverhältnis. Eine Rückgewähr von
  Beiträgen, Spenden oder sonstigen Unterstützungsleistungen ist
  grundsätzlich ausgeschlossen.
\item
  Der Anspruch des Vereins auf rückständige Beitragsforderungen bleibt
  von einem Ende der Mitgliedschaft unberührt.
\end{enumerate}

\subsubsection*{§5 Rechte und Pflichten der
Mitglieder}\label{rechte-und-pflichten-der-mitglieder}

\begin{enumerate}
\def\labelenumi{\arabic{enumi}.}
\item
  Die aktiven Mitglieder sind berechtigt, an allen angebotenen
  Veranstaltungen des Vereins teilzunehmen. Sie haben darüber das Recht,
  gegenüber dem Vorstand und der Mitgliederversammlung Anträge zu
  stellen und an Abstimmungen teilzunehmen.
\item
  Fördermitglieder haben das Recht, Vorschläge zu Aktivitäten des
  Vereins zu machen und Informationen über die Verwendung der
  Förderbeiträge zu erhalten.
\item
  Die Mitglieder sind verpflichtet, den Verein und Vereinszweck - auch
  in der Öffentlichkeit - in ordnungsgemäßer Weise zu unterstützen.
\item
  Der Verein erhebt einen Mitgliedsbeitrag, zu dessen Zahlung die
  Mitglieder verpflichtet sind. Näheres regelt eine Beitragsordnung, die
  von der Mitgliederversammlung beschlossen wird.
\end{enumerate}

\subsubsection*{§6 Ausschluss aus dem
Verein}\label{ausschluss-aus-dem-verein}

\begin{enumerate}
\def\labelenumi{\arabic{enumi}.}
\item
  Der Ausschluss eines Mitgliedes mit sofortiger Wirkung und aus
  wichtigem Grund kann dann ausgesprochen werden, wenn das Mitglied in
  grober Weise dem Zwecke, der Satzung, den Zielen oder der Ordnung des
  Vereins zuwider handelt oder das Ansehen des Vereins in der
  Öffentlichkeit in grober Weise schädigt.
\item
  Über den Ausschluss entscheidet der Vorstand in einfacher
  Stimmmehrheit.
\item
  Dem Mitglied ist unter Fristsetzung von zwei Wochen Gelegenheit zu
  geben, sich vor dem Vereinsausschluss zu den Vorwürfen zu äußern. Legt
  das Mitglied gegen den Ausschluss Widerspruch beim Vorstand ein, so
  entscheidet die Mitgliederversammlung endgültig über den Ausschluss.
\end{enumerate}

\subsubsection*{§7 Organe des Vereins}\label{organe-des-vereins}

\begin{enumerate}
\def\labelenumi{\arabic{enumi}.}
\tightlist
\item
  Organe des Vereins sind
\end{enumerate}

\begin{itemize}
\tightlist
\item
  die Mitgliederversammlung;
\item
  der Vorstand.
\end{itemize}

\begin{enumerate}
\def\labelenumi{\arabic{enumi}.}
\setcounter{enumi}{1}
\tightlist
\item
  Einem Organ des Vereins können nur aktive Mitglieder des Vereins
  angehören.
\end{enumerate}

\subsubsection*{§8 Die
Mitgliederversammlung}\label{die-mitgliederversammlung}

\begin{enumerate}
\def\labelenumi{\arabic{enumi}.}
\item
  Die Mitgliederversammlung ist das oberste Beschlussorgan des Vereins.
  Ihr obliegen alle Entscheidungen, die nicht durch die Satzung oder die
  Beitrags- beziehungsweise Geschäftsordnung einem anderen Organ
  übertragen wurden.
\item
  Beschlüsse werden von der Mitgliederversammlung durch öffentliche
  Abstimmung getroffen. Auf Wunsch eines Mitglieds ist geheim
  abzustimmen.
\item
  Eine ordentliche Mitgliederversammlung wird vom Vorstand des Vereins
  nach Bedarf, mindestens aber einmal im Geschäftsjahr, einberufen. Die
  Einladung erfolgt 14 Tage vorher schriftlich durch den Vorstand mit
  Bekanntgabe der vorläufig festgesetzten Tagesordnung an die dem Verein
  zuletzt bekannten Adresse des Mitglieds.
\item
  Der Vorstand hat eine außerordentliche Mitgliederversammlung
  unverzüglich einzuberufen, wenn die Einberufung von mindestens einem
  Drittel der stimmberechtigten Vereinsmitglieder schriftlich unter
  Angabe des Zwecks und der Gründe vom Vorstand verlangt wird.
\item
  Anträge der Mitglieder zur Tagesordnung sind spätestens eine Woche vor
  der Mitgliederversammlung beim Vorstand schriftlich einzureichen.
  Nachträglich eingereichte Tagesordnungspunkte müssen den Mitgliedern
  rechtzeitig vor Beginn der Mitgliederversammlung mitgeteilt werden.
\item
  Spätere Anträge - auch während der Mitgliederversammlung gestellte
  Anträge - müssen auf die Tagesordnung gesetzt werden, wenn in der
  Mitgliederversammlung die Mehrheit der erschienenen stimmberechtigten
  Mitglieder der Behandlung der Anträge zustimmt
  (Dringlichkeitsanträge).
\item
  Der Vorsitzende oder sein Stellvertreter leitet die
  Mitgliederversammlung. Auf Vorschlag des Vorsitzenden kann die
  Mitgliederversammlung einen besonderen Versammlungsleiter bestimmen.
\item
  Beschlüsse der Mitgliederversammlung werden in einem Protokoll
  innerhalb von zwei Wochen nach der Mitgliederversammlung niedergelegt
  und von zwei Vorstandsmitgliedern unterzeichnet. Abschriften des
  Protokolls werden den Mitgliedern schriftlich zugestellt.
\item
  Stimmberechtigt sind alle aktiven Mitglieder des Vereins. Jedes
  stimmberechtigte Mitglied hat genau eine Stimme, die nur persönlich
  ausgeübt werden darf.
\item
  Die Mitgliederversammlung ist beschlussfähig, wenn die Einladung
  ordnungsgemäß erfolgt ist und bei der Mitgliederversammlung mehr als
  ein Viertel der aktiven Mitglieder des Vereins, mindestens aber fünf
  aktive Mitglieder anwesend sind. Im Falle einer nicht beschlussfähigen
  Mitgliederversammlung ist durch den Vorstand innerhalb von zwei Wochen
  eine Mitgliederversammlung ordnungsgemäß einzuberufen, die in jedem
  Fall beschlussfähig ist. Dies muss in der Einladung explizit vermerkt
  sein.
\item
  Die Mitgliederversammlung fasst ihre Beschlüsse mit einfacher
  Mehrheit. Stimmenthaltungen bleiben außer Betracht. Bei
  Stimmgleichheit gilt der gestellte Antrag als abgelehnt.
\item
  Abstimmmungen der Mitgliederversammlungen erfolgen offen durch
  Handhaufhaben oder Zuruf.
\end{enumerate}

\subsubsection*{§9 Satzungs- und
Ordnungsänderungen}\label{satzungs--und-ordnungsuxe4nderungen}

\begin{enumerate}
\def\labelenumi{\arabic{enumi}.}
\item
  Über Satzungs- und Beitragsordnungsänderungen kann die
  Mitgliederversammlung abstimmen, wenn auf diesen Tagesordnungspunkt
  hingewiesen wurde und der Einladung sowohl der bisherige als auch der
  vorgesehene neue Text beigefügt wurden.
\item
  Für die Satzungs- und Ordnungsänderungen ist eine Mehrheit von zwei
  Dritteln in der Mitgliederversammlung erforderlich.
\item
  Satzungsänderungen, die von Aufsichts-, Gerichts- oder Finanzbehörden
  aus formalen Gründen verlangt werden, kann der Vorstand von sich aus
  vornehmen. Diese Satzungsänderungen müssen auf der nächsten
  Mitgliederversammlung mitgeteilt werden.
\end{enumerate}

\subsubsection*{§10 Der Vorstand}\label{der-vorstand}

\begin{enumerate}
\item
  Der Vorstand setzt sich zusammen aus dem Vorsitzenden, seinem
  Stellvertreter und dem Schatzmeister.
\item
  Die Vorstandsmitglieder werden von der Mitgliederversammlung aus den
  aktiven Mitgliedern für die Dauer von einem Jahr gewählt. Die
  unbegrenzte Wiederwahl von Vorstandsmitgliedern ist zulässig. Nach
  Fristablauf bleiben die Vorstandsmitglieder bis zum Antritt ihrer
  Nachfolger im Amt.
\item
  Der Vorstand leitet verantwortlich die Vereinsarbeit. Er kann sich
  eine Geschäftsordnung geben und besondere Aufgaben unter seinen
  Mitgliedern verteilen oder Ausschüsse für die Bearbeitung oder
  Vorbereitung einsetzen.
\item
  Der Verein wird im Sinne des §26 BGB vom Vorsitzenden und von seinem
  Stellvertreter jeweils einzeln gerichtlich und außergerichtlich
  vertreten. Der Vorsitzende und sein Stellvertreter sind von den
  Beschränkungen des §181 BGB befreit.
\end{enumerate}
\newpage
\begin{enumerate}
\setcounter{enumi}{3}
\item[4a.] 
  Die folgenden Beschlüsse können nicht durch den Vorstand gefällt
  werden: \\
  - Entscheidungen zur Kündigung oder Aufnahme eines Mietverhältnisses für die Räumlichkeiten des Vereins. \\\\
  Soll die Mitgliederversammlung über einen der oben genannten Punkte entscheiden,
  so ist das auf der Einladung als Tagesordnungspunkt kenntlich zu machen.
  
\end{enumerate}

\begin{enumerate}
\setcounter{enumi}{4}
\item
  Der Vorstand beschließt mit einfacher Stimmmehrheit. Der Vorstand ist
  beschlussfähig, wenn mindestens zwei Mitglieder anwesend sind oder
  schriftlich zustimmen. Bei Stimmgleichheit gilt der Antrag als
  abgelehnt. Beschlüsse des Vorstands werden in einem Sitzungsprotokoll
  niedergelegt und von mindestens zwei Vorstandsmitgliedern
  unterzeichnet.
\item
  Scheidet ein Vorstandsmitglied vor Ablauf seiner Wahlzeit aus, ist der
  Vorstand berechtigt, ein kommissarisches Mitglied zu berufen. Auf
  diese Weise bestimmte Vorstandsmitglieder bleiben bis zur nächsten
  Mitgliederversammlung im Amt.
\item
  Der Vorsitzende zeichnet verantwortlich für die Berichtsfähigkeit
  gegenüber der Aufsichtsbehörde.
\item
  Die Vorstandsmitglieder sind grundsätzlich ehrenamtlich tätig; sie
  haben Anspruch auf Erstattung notwendiger Auslagen, sofern deren
  Rahmen in der Geschäftsordnung festgelegt wird.
\item
  Die Mitgliederversammlung stimmt über die Entlastung des Vorstands ab.
\end{enumerate}

\subsubsection*{§11 Kassenprüfer}\label{kassenpruxfcfer}

\begin{enumerate}
\def\labelenumi{\arabic{enumi}.}
\item
  Über die Jahresmitgliederversammlung sind zwei Kassenprüfer für die
  Dauer von einem Jahr zu wählen. Diese dürfen nicht Mitglied des
  Vorstands oder eines vom Vorstand berufenen Gremiums sein.
\item
  Die Kassenprüfer haben die Aufgabe, Rechnungsbelege sowie deren
  ordnungsgemäße Verbuchung und die Mittelverwendung zu prüfen und dabei
  insbesondere die satzungsgemäße und steuerlich korrekte
  Mittelverwendung festzustellen. Die Prüfung erstreckt sich nicht auf
  die Zweckmäßigkeit der vom Vorstand getätigten Ausgaben.
\item
  Die Kassenprüfer haben die Mitgliederversammlung über das Ergebnis
  ihrer Kassenprüfung zu unterrichten und werden auf Grundlage des
  Berichts von der Mitgliederversammlung entlastet.
\end{enumerate}

\subsubsection*{§12 Veranstaltungen und
Projekte}\label{veranstaltungen-und-projekte}

\begin{enumerate}
\def\labelenumi{\arabic{enumi}.}
\item
  Der Verein darf von den Teilnehmern einer Veranstaltung oder eines
  Projekts einen jeweils vom Vorstand festzusetzenden Unkostenbeitrag
  erheben.
\item
  Die Veranstaltungen und Projekte dürfen von Personen geleitet werden,
  die nicht dem Verein angehören. Die Leitung darf eine gewerbliche
  Dienstleistung sein, die aus der Vereinskasse vergütet wird.
\end{enumerate}

\subsubsection*{§13 Übertragung von
Aufgaben}\label{uxfcbertragung-von-aufgaben}

\begin{enumerate}
\def\labelenumi{\arabic{enumi}.}
\item
  Der Vorstand ist berechtigt, bestimmte Aufgaben zur Erfüllung des
  Vereinszwecks auf andere Personen, insbesondere auf Veranstaltungs-
  und Projektleiter zu übertragen.
\item
  Die Übertragung darf schriftlich oder mündlich erfolgen.
\end{enumerate}

\subsubsection*{§14 Auflösung des
Vereins}\label{aufluxf6sung-des-vereins}

\begin{enumerate}
\def\labelenumi{\arabic{enumi}.}
\item
  Die Auflösung des Vereins muss von der Mitgliederversammlung mit einer
  Mehrheit von zwei Dritteln beschlossen werden.
\item
  Die Abstimmung ist nur möglich, wenn auf der Einladung zur
  Mitgliederversammlung als einziger Tagesordnungspunkt die Auflösung
  des Vereins angekündigt wurde.
\item
  Bei Auflösung des Vereins oder bei Wegfall seiner steuerbegünstigten
  Zwecke fällt das Vereinsvermögen an digitalcourage e.V., der es
  ausschließlich und unmittelbar zu steuerbegünstigten Zwecken zu
  verwenden hat.
\end{enumerate}

\subsubsection*{§15 Mitgliedschaft in anderen
Vereinen}\label{mitgliedschaft-in-anderen-vereinen}

\begin{enumerate}
\def\labelenumi{\arabic{enumi}.}
\item
  Der Verein darf Mitglied in anderen Vereinen werden.
\item
  Die Mitgliederversammlung entscheidet über den Beitritt bzw. Austritt
  aus anderen Vereinen.
\item
  Diese Entscheidung kann auch vom Vorstand gefällt werden und muss in
  der nächsten Mitgliederversammlung bestätigt werden. Falls die
  Mitgliederversammlung gegen die vom Vorstand getroffene Entscheidung
  stimmt, erfolgt der Austritt aus dem Verein zum nächstmöglichen
  Termin.
\end{enumerate}

\subsubsection*{§16 Schriftform}\label{schriftform}

\begin{enumerate}
\def\labelenumi{\arabic{enumi}.}
\item
  Schriftliche Erklärungen im Sinne dieser Satzung können auch
  elektronische Dokumente sein.
\item
  Die Geschäftsordnung bestimmt Anforderungen, Zustellwege und Zuordnung
  derartiger Dokumente.
\end{enumerate}

\end{document}
